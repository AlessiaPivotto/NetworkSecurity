\chapter{Network aspects}
Internet communication 
Internet is made of several logically separated networks -> Autonomous Systems (AS)
PS: Internet= network of networks
Each AS autonomously manages communications within itself through Interior Gateway Protocols (IGP) -> route within each AS
e.g. two commonly used IGPs are the Routing Information Protocol (RIP) and the Open Shortest Path First (OSPF) protocol.
Each AS can communicate to other AS using Exterior Gateways Protocols -> route between ASs, such as Border Gateway Protocol.


OSI data link layer
Data link layer is the lowest “logical” level, data link interconnects physical interfaces and each physical interface is identified by a MAC address(“Ethernet address”, 48-bit Network interface identifiers, closest representation of final destination of a frame, HEX notation, used to route packets in local networks).



Mac addresses uniquely identify a network interface and is assigned by the producer according to the standard IEEE 802. 
eg da fare 

OSI NETWORK LAYER
The Network Layer provides information on how to reach other systems (addressing functionalities), IP operates at this layer so high-level representation of a host’s addresses, conveys information to route the datagram, IPv4 defined in RFC 791
Most IP addresses are dynamically assigned by an authority (e.g. ISP’s DHCP server). As opposed to MAC addresses that are fixed by the vendor. “Connectionless” protocol (stateless): no notion of “established connection” at this stage, only provides the means necessary for a packet to reach its destination.

connection vs connectionless

A communication is made of a number of messages; communications start, develop, and end. Stateful protocols provide means to establish and close a connection(e.g. TCP).
Stateless protocols do not have this notion(IP messages are stand-alone packets).

ip vs mac da fare

IP addresses
IP provides a structured way to abstract host addresses away from their physical properties
There are two versions: IPv4 and IPv6, the first is the most common and is currently used,(32 bits) while the latter is early adoption and will be seen commonly in the future(128 bits).
Make it possible to efficiently talk between systems in different AS

routing da fare 

ARP protocol
ARP = address resolution protocol
Allows systems to associate an IP address to a MAC address and allows discovery through broadcast. ARP tables contain information to translate IP addresses into MAC addresses.

arp tables da fare 

ARP query

All addresses in an ARP table are added by one of two mechanisms:
– ARP request-reply
-> who is 192.168.0.16 tells 192.168.0.1
-> 192.168.0.16 is at 00-10-BC-2c-11-56
– Gratuitous ARP
-> 192.168.0.16 is at 00-10-BC-2c-11-56

The discovery process happens through queries to neighbor devices
– Broadcast message to the desired IP
• L2 ethernet address FF-FF-FF-FF-FF-FF

The system with the requested IP replies back with its correct mac address


frame + tables + eg+ broadcast ancora da fare 

ARP poisoning

ARP answers or Gratuitous ARP frames do not require an (additional) answer/confirmation, so it’s a declarative protocol. This means that nodes are not authenticated, so whomever can say I am x.x1.x2.x3, my mac address is hh.hh1.hh2.hh3.hh4.hh5.
• C can tell B “D is at [C mac address]”
• C can tell D “B is at [C mac address]”
• As a result every communication between B and D will pass by C

da finire

ARP poisoning - limitations
Works only on local networks, where MAC addresses are actually meaningful. 
NB: when communication is targeted to different networks, IP addresses are used.
Routers and DNSs have MAC addresses too..
The poisoning works because systems are not authenticated, some implementations/third party tools can mitigate the problem, such as checking for anomalies.

ip header credo convenga guardare da doriguzzi


Subnets and CIDR + es da fare 

Subnets are logical divisions of IP addresses, so possible to split a network in multiple sub-networks
IP bits are divided in:
– x network bits
– y subnet bits
– z host bits
Subnet mask indicates sections of IP addresses meant for network+subnet
– 255.255.255.0 à 24 bits to network+subnet, 8 bits to hosts
CIDR à synthetic way to represent subnet masks
– Classless Inter-Domain Routing
– Indicates number of bits covered by the mask
– 192.168.10.1/24 = 192.168.10.1/255.255.255.0
